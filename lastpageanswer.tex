\documentclass{article}
\usepackage[utf8]{inputenc}
\begin{document}
\section*{MCPS Surgery Revision Flashcards}


This document answers common short-note questions from \textbf{Bailey \& Love's Short Practice of Surgery} and other standard texts. Each entry provides a brief introduction, detailed explanation, and a concluding remark to help consolidate knowledge for exam revision.


\section{Pneumo-peritoneum in laparoscopy (January, 2009)}


\textbf{Question:} Describe pneumoperitoneum in laparoscopy.

\textbf{Answer}

\emph{Introduction:} Pneumoperitoneum is created by insufflating gas into the abdominal cavity to provide working space for minimally invasive surgery. Bailey \& Love describes it as the fundamental first step that allows visualization of intra-abdominal organs while minimizing trauma. Carbon dioxide is the preferred gas because it is highly soluble in blood, non-combustible, and easily removed by ventilation.

\emph{Main Content:} Establishment of pneumoperitoneum can be achieved using a Veress needle or an open technique such as Hasson entry. The initial insufflation pressure is usually set between 12 and 15 mmHg to create adequate space while limiting cardiovascular compromise. Higher pressures can impede venous return, elevate systemic vascular resistance, and increase airway pressure, so anesthetic monitoring is crucial. Surgeons must be cautious to avoid visceral or vascular injury when placing the first trocar. Complications include subcutaneous emphysema, gas embolism, and unrecognized bowel perforation. Patients with previous abdominal surgery may require optical trocars or an open approach due to adhesions. Physiological effects extend beyond the cardiovascular system; elevated intra-abdominal pressure reduces renal perfusion and affects diaphragmatic motion, potentially increasing the risk of postoperative atelectasis. Alternative methods such as gasless laparoscopy using abdominal wall lifting devices exist, but they are less common. Modern insufflators can regulate pressure and flow more precisely and incorporate smoke evacuation to maintain visibility. Appropriate patient positioning—Trendelenburg or reverse Trendelenburg—helps optimize exposure but can further alter venous return and respiratory mechanics.

\emph{Conclusion:} Proficiency in establishing and managing pneumoperitoneum is essential for safe laparoscopy. Understanding its physiological impact, carefully selecting the access technique, and close coordination with anesthesia minimize risks. When employed correctly, pneumoperitoneum enables surgeons to perform a wide range of procedures through small incisions, resulting in less postoperative pain, shorter hospitalization, and quicker recovery compared with traditional open surgery.


\section{Bipolar diathermy (January, 2009)}


\textbf{Question:} What is bipolar diathermy and how is it used?

\textbf{Answer}

\emph{Introduction:} Bipolar diathermy utilizes an electrosurgical forceps through which current passes between two closely spaced tips. Unlike the monopolar system where current travels through the patient to a return plate, bipolar current is confined to tissue held between the prongs, allowing precise coagulation. Bailey \& Love recommends bipolar devices when operating near delicate structures because the limited current path minimizes unintended thermal injury.

\emph{Main Content:} The technology works by generating radiofrequency energy that causes ionic agitation and heating of tissue, leading to protein denaturation and vessel sealing. It is particularly useful in neurosurgery, ophthalmic surgery, and procedures around major nerves or vessels where precision is paramount. Because the current does not traverse the whole body, there is no need for a grounding pad, reducing the risk of burns at remote sites. Surgeons must ensure the tips are clean and dry to obtain efficient coagulation. Settings vary depending on tissue type—higher power may be required for fatty tissue compared with vascular structures. Modern bipolar units may incorporate feedback systems to deliver just enough energy to achieve hemostasis without excessive charring. Complications are rare but can include thermal spread if high power is used or if activation is prolonged. The surgeon should avoid activating the instrument near flammable gases, as with all electrosurgery. Regular equipment checks and maintenance are essential, including inspecting insulation to prevent stray currents. Foot pedal or hand controls allow momentary activation, giving surgeons tactile feedback and improved safety.

\emph{Conclusion:} Bipolar diathermy provides precise, localized coagulation with reduced risk of collateral damage compared to monopolar techniques. When used with proper settings and attention to safety protocols, it is an invaluable tool for achieving hemostasis in areas where protecting surrounding tissue is critical.


\section{Disseminated Intravascular Coagulation (January, 2009)}


\textbf{Question:} What is disseminated intravascular coagulation and how is it managed?

\textbf{Answer}

\emph{Introduction:} Disseminated intravascular coagulation (DIC) is a pathological process in which widespread activation of the coagulation cascade leads to intravascular fibrin deposition and consumption of platelets and clotting factors. Bailey \& Love emphasizes that DIC is not a primary disease but a complication of severe conditions such as trauma, sepsis, or obstetric catastrophe. Early recognition is vital because DIC can rapidly progress to multiorgan failure.

\emph{Main Content:} The underlying trigger for DIC is the release of procoagulant substances into the bloodstream, often due to tissue damage or infection. This causes extensive microvascular thrombosis, leading to ischemia and organ dysfunction. Concurrently, the consumption of clotting factors and platelets results in bleeding from surgical sites, mucous membranes, or puncture wounds. Laboratory findings include prolonged prothrombin time, activated partial thromboplastin time, low fibrinogen levels, and elevated D-dimer. Management begins with aggressive treatment of the precipitating condition: controlling infection, draining abscesses, or removing necrotic tissue. Supportive care includes maintaining hemodynamic stability, optimizing oxygenation, and replacing depleted coagulation factors with fresh-frozen plasma and platelets. Cryoprecipitate may be required for very low fibrinogen levels. Heparin therapy is controversial; it may be considered in cases dominated by thrombosis rather than bleeding, but its use should be individualized. Monitoring trends in laboratory parameters guides therapy. In critically ill patients, admission to intensive care allows continuous observation of organ function. Replacement therapy should be titrated to clinical and laboratory response.

\emph{Conclusion:} DIC requires prompt diagnosis and a multifaceted approach targeting the underlying cause while supporting coagulation. Timely intervention can prevent catastrophic bleeding or irreversible organ damage. A high index of suspicion, particularly in high-risk settings like trauma and sepsis, allows surgeons to initiate treatment early and improve patient outcomes.


\section{Pressure sore (July, 2008)}


\textbf{Question:} Discuss pressure sores and their management.

\textbf{Answer}

\emph{Introduction:} Pressure sores, also called decubitus ulcers, arise when prolonged pressure over bony prominences leads to ischemia and subsequent tissue breakdown. They are common in bedridden or neurologically impaired patients. Bailey \& Love highlights the importance of prevention, as established ulcers can be difficult to heal and may predispose patients to infection and prolonged hospitalization.

\emph{Main Content:} The pathogenesis involves sustained external pressure exceeding capillary perfusion pressure, leading to local ischemia. Shear forces, friction, and moisture from incontinence exacerbate tissue damage. Staging ranges from nonblanching erythema to deep ulcers exposing muscle or bone. Management begins with identifying at-risk patients, such as those with spinal cord injury, advanced age, or malnutrition. Preventive measures include frequent repositioning—ideally every two hours—use of pressure-relieving mattresses, and meticulous skin care. Once a sore develops, treatment depends on depth. Superficial ulcers may respond to offloading pressure, absorbent dressings, and topical agents. Deeper wounds often require debridement of necrotic tissue, management of infection with antibiotics, and optimization of nutrition, particularly protein and vitamins. Vacuum-assisted closure can help promote granulation. Surgical options, such as flap coverage, may be considered for chronic stage III or IV ulcers when conservative measures fail. Multidisciplinary involvement of nursing staff, physiotherapists, dietitians, and wound specialists is essential. Regular assessment with wound measurements and photographic documentation monitors progress. Pain control should not be neglected, as ulcer care can be uncomfortable.

\emph{Conclusion:} Prevention remains the cornerstone in managing pressure sores. Early recognition of risk factors, diligent nursing care, and appropriate use of support surfaces can avert these often devastating lesions. When ulcers do occur, a structured approach that addresses pressure relief, wound care, infection control, and nutritional support offers the best chance of healing and reduces the risk of recurrence.


\section{Multiple discharging sinus in axilla (July, 2008)}


\textbf{Question:} Explain the causes and treatment of multiple discharging sinuses in the axilla.

\textbf{Answer}

\emph{Introduction:} Multiple sinuses in the axillary region are most often caused by chronic inflammatory disorders of apocrine sweat glands, particularly hidradenitis suppurativa. Bailey \& Love notes that early diagnosis and appropriate management are necessary to prevent extensive scarring and functional limitation in this anatomically complex area.

\emph{Main Content:} Hidradenitis suppurativa begins with occlusion of hair follicles, leading to recurrent abscess formation. With repeated inflammation, sinus tracts develop and may connect beneath the skin, creating multiple discharging openings. Contributing factors include obesity, smoking, tight clothing, and sometimes a genetic predisposition. Acute flares may be managed with antibiotics such as tetracycline or clindamycin, combined with improved hygiene and avoidance of irritants. Chronic disease with sinus formation often requires surgical intervention. Limited excision of sinus tracts can be effective in early stages, while wide local excision of affected skin and subcutaneous tissue may be indicated for advanced disease. Reconstruction can involve split-thickness skin grafts or local flaps. Newer treatments, including laser hair removal, biological agents, and lifestyle modifications such as weight loss and smoking cessation, may reduce recurrence. It is important to differentiate hidradenitis from other causes of axillary sinuses, such as tuberculosis, actinomycosis, or metastatic lymph node disease, which require specific therapy. Histological examination of excised tissue helps rule out malignancy that can rarely arise in chronic lesions.

\emph{Conclusion:} Multiple axillary sinuses are most commonly due to hidradenitis suppurativa. Effective treatment combines medical therapy for acute inflammation with surgical excision for chronic, scarred disease. Early recognition and comprehensive management can significantly improve quality of life and limit disability.


\section{Necrotizing fasciitis (July, 2008)}


\textbf{Question:} What is necrotizing fasciitis and how is it managed?

\textbf{Answer}

\emph{Introduction:} Necrotizing fasciitis is a rapidly progressive infection characterized by widespread necrosis of the fascia and subcutaneous tissues. Bailey \& Love identifies it as a surgical emergency that can quickly lead to systemic toxicity and death if not promptly diagnosed and treated. Early symptoms may be deceptively mild, making vigilance essential.

\emph{Main Content:} The condition is often classified into Type I, a polymicrobial infection involving aerobic and anaerobic bacteria, and Type II, typically caused by group A Streptococcus. Predisposing factors include diabetes, immunosuppression, chronic kidney disease, and minor trauma or surgery. Clinical features include severe pain disproportionate to physical findings, swelling, erythema, and skin discoloration. Crepitus may be present if gas-forming organisms are involved. Imaging with CT or MRI can reveal fascial thickening and gas, aiding diagnosis, but treatment should not be delayed for imaging if clinical suspicion is high. Management demands immediate, aggressive surgical debridement of all necrotic tissue, often requiring repeated operations. Broad-spectrum intravenous antibiotics covering Gram-positive, Gram-negative, and anaerobic organisms are essential; regimens may include penicillin, clindamycin, and a third-generation cephalosporin. Supportive care in an intensive care unit addresses septic shock and organ dysfunction. Adjunctive therapies, such as hyperbaric oxygen and intravenous immunoglobulin, have been used but with variable evidence. After infection control, reconstructive surgery may be needed for extensive tissue loss.

\emph{Conclusion:} Rapid recognition and prompt surgical intervention, combined with broad-spectrum antibiotics and intensive support, are vital to survival in necrotizing fasciitis. Delays in debridement substantially increase mortality, underscoring the need for high clinical suspicion and decisive management.


\section{Sentinel node (July, 2008)}


\textbf{Question:} Define a sentinel lymph node and its role in cancer surgery.

\textbf{Answer}

\emph{Introduction:} A sentinel lymph node is the initial lymph node or group of nodes that receives lymphatic drainage from a primary tumor site. The concept, highlighted in both Bailey \& Love and modern oncologic textbooks, underpins the minimally invasive approach to staging many cancers. By analyzing the sentinel node, surgeons can determine whether metastasis has occurred without performing extensive nodal dissection.

\emph{Main Content:} Sentinel node mapping is commonly used in breast cancer and melanoma surgery but has applications in other cancers, such as vulvar carcinoma. The procedure involves injecting a tracer—either a blue dye, a radioactive colloid, or both—around the tumor or biopsy site. Lymphatic channels transport the tracer to the sentinel node, which is identified intraoperatively with a gamma probe or by visualizing blue staining. This node is removed and sent for immediate pathological examination using frozen section or subsequently with standard histology and immunohistochemistry. If metastatic cells are detected, further nodal clearance or adjuvant therapy may be indicated. Conversely, if the sentinel node is negative, the patient can avoid the morbidity associated with complete lymph node dissection, such as lymphedema, nerve injury, and prolonged wound healing. Accuracy depends on meticulous technique and anatomical understanding of lymphatic drainage patterns. False negatives can occur if the tracer bypasses a diseased or obstructed node or if multiple lymphatic basins are involved. Training and multidisciplinary cooperation between surgeons, nuclear medicine physicians, and pathologists enhance success rates.

\emph{Conclusion:} Sentinel node biopsy is a valuable tool for staging and guiding treatment in many cancers. By focusing on the first draining node, surgeons can obtain crucial prognostic information while sparing patients the complications of more radical surgery when it is not necessary.


\section{Problems of surgery in an obese patient (January, 2008)}


\textbf{Question:} Enumerate the problems associated with operating on an obese patient.

\textbf{Answer}

\emph{Introduction:} Obesity is a growing health issue worldwide and presents unique challenges in the perioperative period. Bailey \& Love notes that obesity complicates nearly every aspect of surgical care, from airway management to postoperative recovery. Understanding these challenges allows the surgical team to plan appropriately and mitigate risks.

\emph{Main Content:} Preoperatively, obese patients often have associated comorbidities such as diabetes, hypertension, and obstructive sleep apnea, which increase perioperative risk. Anesthetic induction may be difficult due to limited neck mobility and excess soft tissue, making airway management challenging. Venous access can be problematic, and patient positioning requires additional care to prevent nerve compression and pressure injuries. Intraoperatively, thick adipose layers obscure anatomical landmarks, prolonging surgical time and potentially increasing blood loss. Laparoscopic procedures may require higher insufflation pressures and longer instruments. Obesity also predisposes to thromboembolic events, so careful prophylaxis with compression stockings and pharmacologic anticoagulation is essential. Postoperatively, wound complications such as infection, dehiscence, and incisional hernia are more common due to poor vascularity of adipose tissue and increased tension on suture lines. Respiratory complications are heightened because decreased lung volumes and obstructive sleep apnea lead to hypoventilation and atelectasis. Mobilization is often delayed, exacerbating risk of deep vein thrombosis and pulmonary embolism. Dosing of medications, particularly anesthetics and antibiotics, must consider ideal versus total body weight. Specialized bariatric equipment, including wide operating tables, longer retractors, and imaging devices, may be required.
\section{Surgical ethics (January, 2008)}


\textbf{Question:} Summarize the key principles of surgical ethics.

\textbf{Answer}

\emph{Introduction:} Ethics in surgery guide decisions that affect patient wellbeing and professional integrity. Bailey \& Love outlines fundamental principles that surgeons must uphold in daily practice.

\emph{Main Content:} The four pillars of medical ethics—autonomy, beneficence, non-maleficence, and justice—apply directly to surgery. Autonomy requires obtaining informed consent and respecting patient choices. Beneficence and non-maleficence obligate surgeons to act in the patient's best interest while avoiding unnecessary harm. Justice entails fair allocation of healthcare resources and unbiased treatment of all patients. Additional concerns include maintaining confidentiality, providing accurate information, and disclosing potential conflicts of interest. Surgeons must also adhere to professional guidelines, continuing education, and responsible research conduct.

\emph{Conclusion:} Adherence to ethical principles fosters trust, ensures patient safety, and underpins the reputation of the surgical profession.


\section{Pain palliation in cancer patient (January, 2008)}


\textbf{Question:} Describe the approach to pain palliation in a cancer patient.

\textbf{Answer}

\emph{Introduction:} Effective pain control improves quality of life for cancer patients and is an integral part of oncologic management. Bailey \& Love recommends a stepwise approach tailored to individual needs.

\emph{Main Content:} The World Health Organization analgesic ladder guides pharmacologic therapy, beginning with non-opioid analgesics like paracetamol or NSAIDs. If pain persists, mild opioids such as codeine are added, progressing to strong opioids like morphine for severe pain. Adjuvant medications, including antidepressants or anticonvulsants, help in neuropathic pain. For refractory cases, interventions such as nerve blocks, epidural infusions, or spinal pumps may be considered. Non-pharmacologic measures—psychological support, physiotherapy, and complementary therapies—also play roles. Regular assessment of pain scores ensures adequate relief and helps adjust therapy.

\emph{Conclusion:} A systematic, multidisciplinary approach is crucial for optimal pain palliation, respecting patient preferences and balancing efficacy with side effects.


\section{Post operative pulmonary complications (July, 2007)}


\textbf{Question:} List common postoperative pulmonary complications and preventive measures.

\textbf{Answer}

\emph{Introduction:} Pulmonary complications remain a significant source of morbidity following surgery, particularly thoracic or upper abdominal procedures.

\emph{Main Content:} Common problems include atelectasis, pneumonia, bronchospasm, and exacerbation of chronic lung disease. Risk factors include smoking, obesity, advanced age, and prolonged anesthesia. Prevention begins preoperatively by optimizing pulmonary function and encouraging smoking cessation. Intraoperatively, careful ventilation strategies and limiting operative time reduce risk. Postoperatively, early mobilization, deep breathing exercises, incentive spirometry, and adequate analgesia are vital to maintain lung expansion. Chest physiotherapy and appropriate antibiotic use can manage or prevent pneumonia. Regular monitoring of oxygen saturation and respiratory rate allows prompt detection of deterioration.

\emph{Conclusion:} A proactive approach combining preoperative optimization and vigilant postoperative care is essential to minimize pulmonary complications.


\section{ARDS (July, 2007)}


\textbf{Question:} Outline acute respiratory distress syndrome (ARDS) and its management.

\textbf{Answer}

\emph{Introduction:} ARDS is a severe inflammatory lung injury characterized by acute onset of hypoxemia and non-cardiogenic pulmonary edema. Bailey \& Love describes it as a critical care challenge often following sepsis or trauma.

\emph{Main Content:} Diagnostic criteria include acute onset within one week of an insult, bilateral pulmonary infiltrates on imaging, and PaO2/FiO2 ratio less than 300 despite adequate PEEP. Management focuses on treating the underlying cause and providing supportive ventilation. Low tidal volume ventilation (6 mL/kg of ideal body weight) prevents ventilator-induced lung injury, while higher PEEP improves oxygenation. Fluid management should avoid overload. Adjuncts may include prone positioning, neuromuscular blockade, or extracorporeal membrane oxygenation in severe cases. Mortality remains high despite advances.

\emph{Conclusion:} Early recognition and lung-protective ventilation strategies are central to improving outcomes in ARDS patients.


\section{Rational use of antibiotics (July, 2007)}


\textbf{Question:} Describe the principles of rational antibiotic use.

\textbf{Answer}

\emph{Introduction:} Judicious antibiotic use reduces resistance, side effects, and costs. Surgical texts stress tailoring therapy to the likely organisms and local patterns.

\emph{Main Content:} Key steps include obtaining cultures before starting treatment and choosing narrow-spectrum agents whenever possible. Dosages should be appropriate for the patient's weight and renal function. Duration of therapy should be limited to the minimum effective course to avoid promoting resistant strains. Prophylactic antibiotics for surgery must be given within an hour of incision and stopped within 24 hours unless otherwise indicated. Monitoring for adverse reactions or superinfections is essential.

\emph{Conclusion:} Following evidence-based guidelines for antibiotic selection and duration preserves effectiveness and safeguards patient health.


\section{ARDS (July, 2007)}


\textbf{Question:} Outline acute respiratory distress syndrome (ARDS) and its management.

\textbf{Answer}

\emph{Introduction:} ARDS is a serious form of respiratory failure often triggered by sepsis, trauma, or pancreatitis.

\emph{Main Content:} Diagnostic features and management mirror those described in Question 12. Supportive ventilation with low tidal volumes and appropriate PEEP is essential, along with aggressive treatment of the precipitating illness. Fluid balance should be carefully controlled, and adjunct therapies such as prone positioning or ECMO may be used in refractory cases.

\emph{Conclusion:} The principles of early recognition and lung-protective strategies remain central to improving survival in ARDS.


\section{Rational use of antibiotics (July, 2007)}


\textbf{Question:} Summarize principles for responsible antibiotic therapy.

\textbf{Answer}

\emph{Introduction:} The increase in antimicrobial resistance underscores the need for rational prescribing practices.

\emph{Main Content:} As detailed in Question 13, antibiotics should be chosen based on likely pathogens and local resistance patterns, with culture guidance whenever feasible. Narrow-spectrum agents reduce collateral damage to normal flora. Dosage and duration must suit the individual patient and specific infection. Monitoring for side effects, interactions, and treatment response ensures efficacy. Surgical prophylaxis should be timely and limited in duration.

\emph{Conclusion:} Careful selection and stewardship of antibiotics maintain their effectiveness and reduce complications.


\section{Post operative pulmonary complications (July, 2007)}


\textbf{Question:} Briefly discuss postoperative pulmonary complications and their prevention.

\textbf{Answer}

\emph{Introduction:} Pulmonary complications following surgery can prolong hospital stay and increase mortality.

\emph{Main Content:} As outlined in Question 11, atelectasis and pneumonia are common, especially after upper abdominal surgery. Risk reduction strategies include preoperative smoking cessation, deep breathing exercises, adequate analgesia, early mobilization, and incentive spirometry. Chest physiotherapy and careful fluid management further support lung function.

\emph{Conclusion:} Implementing preventive measures from the preoperative stage through recovery minimizes pulmonary morbidity.


\section{ARDS (July, 2007)}


\textbf{Question:} Summarize acute respiratory distress syndrome.

\textbf{Answer}

\emph{Introduction:} ARDS arises from widespread inflammation in the lungs leading to severe hypoxia.

\emph{Main Content:} As discussed previously in Question 12, management involves identifying and treating the underlying cause, providing lung-protective ventilation, and careful fluid balance. Adjunct therapies like prone positioning or ECMO may be required for refractory hypoxemia.

\emph{Conclusion:} Early and aggressive supportive care offers the best chance for recovery in ARDS.


\section{Rational use of antibiotics (July, 2007)}


\textbf{Question:} Outline measures for rational antibiotic usage.

\textbf{Answer}

\emph{Introduction:} Antimicrobial stewardship limits the emergence of drug resistance and reduces complications.

\emph{Main Content:} Measures echo those in Questions 13 and 15: culture-based selection of agents, appropriate dose and duration, and discontinuation when infection is controlled. Education of healthcare providers and monitoring of resistance patterns guide policy.

\emph{Conclusion:} Thoughtful antibiotic prescribing safeguards patient outcomes and public health.


\section{Limitations of minimal access surgery (January, 2007)}


\textbf{Question:} Discuss the limitations of minimally invasive surgery.

\textbf{Answer}

\emph{Introduction:} Minimal access surgery offers many benefits, but it is not appropriate for all cases. Bailey \& Love emphasizes the importance of patient selection and surgeon experience.

\emph{Main Content:} Limitations include the steep learning curve for complex procedures and the need for expensive equipment and specialized training. Operative times may initially be longer, and tactile feedback is reduced, making some maneuvers difficult. In cases of extensive adhesions, large tumors, or major bleeding, conversion to open surgery may be necessary. Patients with severe cardiopulmonary disease may not tolerate pneumoperitoneum. Furthermore, there are risks of port-site complications such as hernias or trocar injuries.

\emph{Conclusion:} While minimally invasive surgery has revolutionized many operations, understanding its boundaries ensures patient safety and optimal results.


\section{Trophic ulcer (January, 2007)}


\textbf{Question:} What is a trophic ulcer and how is it treated?

\textbf{Answer}

\emph{Introduction:} Trophic ulcers are chronic wounds caused by repeated trauma on an insensate area, commonly seen in patients with peripheral neuropathy.

\emph{Main Content:} They occur due to loss of protective sensation, often from diabetes or leprosy, combined with abnormal pressure or friction. Management focuses on relieving pressure through specialized footwear or orthotic devices, optimizing glycemic control, and treating infection when present. Debridement of necrotic tissue and moist wound dressings aid healing. In resistant cases, surgical procedures such as flap coverage or tendon lengthening may redistribute pressure.

\emph{Conclusion:} Preventive measures and diligent wound care are central to successful management of trophic ulcers.


\section{TPN (January, 2007)}


\textbf{Question:} Define total parenteral nutrition (TPN) and state its indications.

\textbf{Answer}

\emph{Introduction:} Total parenteral nutrition provides all nutritional requirements intravenously when the gastrointestinal tract is non-functional or needs rest.

\emph{Main Content:} Indications include severe pancreatitis, short bowel syndrome, prolonged ileus, or major bowel resection. A central venous catheter delivers a mixture of dextrose, amino acids, lipids, vitamins, and trace elements. Complications include catheter-related infections, metabolic disturbances such as hyperglycemia or electrolyte imbalance, and liver dysfunction with prolonged use. Close monitoring of fluid balance, electrolytes, and liver function tests is essential. When feasible, transitioning to enteral feeding helps maintain gut integrity and reduces complications.

\emph{Conclusion:} TPN is lifesaving when enteral nutrition is impossible but requires meticulous management to avoid serious complications.


\section{Surgical drainage (January, 2007)}


\textbf{Question:} Discuss the principles and methods of surgical drainage.

\textbf{Answer}

\emph{Introduction:} Surgical drains are used to remove collections of fluid or air to prevent infection or promote healing.

\emph{Main Content:} Drains can be open or closed. Open drains like corrugated rubber sheets allow passive egress but increase infection risk. Closed systems, such as Jackson-Pratt or Redivac drains, use negative pressure and are preferable for most procedures. Indications include evacuation of pus from abscesses, seroma prevention after lymph node dissection, or removal of bile or pancreatic leakage. Drains should exit through separate stab incisions and be removed as soon as output decreases to minimize infection risk.

\emph{Conclusion:} Appropriate selection and timely removal of drains help prevent complications and support postoperative recovery.


\section{Safety measures in using electrocautery (January, 2007)}


\textbf{Question:} What safety precautions should be observed when using electrocautery?

\textbf{Answer}

\emph{Introduction:} Electrocautery is ubiquitous in surgery but carries risks such as burns, ignition of flammable gases, and interference with implanted devices.

\emph{Main Content:} Precautions include ensuring a dry operative field and removing flammable materials or gases (e.g., alcohol-based prep solutions). The patient return electrode must be properly positioned for monopolar devices to prevent burns. Equipment should be inspected for insulation failure, and settings should be appropriate for the tissue. In patients with pacemakers, bipolar cautery or short bursts are recommended to minimize interference. Smoke evacuators reduce exposure to surgical plume.

\emph{Conclusion:} Following established safety protocols ensures effective and hazard-free use of electrocautery.


\section{Decubitus ulcer (July, 2006)}


\textbf{Question:} Describe decubitus ulcers and their prevention.

\textbf{Answer}

\emph{Introduction:} Decubitus ulcers, also called pressure sores, result from prolonged pressure causing ischemia of the skin and underlying tissue.

\emph{Main Content:} They typically develop over bony prominences such as the sacrum or heels in bedridden patients. Risk factors include immobility, malnutrition, and moisture from incontinence. Prevention involves frequent repositioning, pressure-relieving mattresses, good nutritional support, and meticulous skin care. Once ulcers occur, management ranges from dressings and debridement to surgical flap coverage in advanced cases. Addressing systemic factors like infection or anemia aids healing.

\emph{Conclusion:} Preventive measures are more effective than treatment, emphasizing the importance of nursing care and patient mobility.


\section{Deep vein thrombosis (July, 2006)}


\textbf{Question:} Outline the prevention and management of deep vein thrombosis (DVT).

\textbf{Answer}

\emph{Introduction:} DVT involves clot formation in deep veins of the legs or pelvis and may lead to pulmonary embolism if untreated.

\emph{Main Content:} Risk factors include immobility, major surgery, trauma, malignancy, and inherited thrombophilia. Prevention strategies comprise early mobilization, graduated compression stockings, and pharmacologic prophylaxis with low-molecular-weight heparin or fondaparinux. Diagnosis is confirmed with duplex ultrasonography. Treatment includes therapeutic anticoagulation for at least three months. Inferior vena cava filters are reserved for cases where anticoagulation is contraindicated or fails. Long-term complications include post-thrombotic syndrome.

\emph{Conclusion:} Vigilant prophylaxis and prompt treatment of DVT reduce morbidity and prevent potentially fatal embolic events.


\section{Frozen section biopsy (July, 2006)}


\textbf{Question:} What is a frozen section biopsy and when is it used?

\textbf{Answer}

\emph{Introduction:} Frozen section is an intraoperative pathological technique that allows rapid diagnosis of tissue specimens.

\emph{Main Content:} During surgery, a small sample of tissue is frozen, sectioned, and examined under a microscope to provide immediate histological information. Uses include confirming malignancy, assessing surgical margins, and evaluating lymph nodes. The main advantage is guiding real-time surgical decisions, such as whether to extend resection or proceed with reconstruction. Limitations include potential artifacts from freezing and limited sensitivity compared with formal paraffin sections. Communication with the pathologist is essential to ensure appropriate specimen handling.

\emph{Conclusion:} Frozen section biopsy aids in immediate decision-making during surgery, enhancing oncologic accuracy.


\section{Endoscopic therapeutic procedures (July, 2006)}


\textbf{Question:} Give examples of therapeutic procedures performed endoscopically.

\textbf{Answer}

\emph{Introduction:} Endoscopy allows diagnosis and treatment of many gastrointestinal and biliary conditions without open surgery.

\emph{Main Content:} Common therapeutic endoscopic procedures include polypectomy for removal of colonic or gastric polyps, endoscopic retrograde cholangiopancreatography (ERCP) with sphincterotomy and stone extraction, endoscopic variceal ligation for portal hypertension, dilation of strictures with balloons, placement of feeding tubes or stents, and management of bleeding ulcers with injection or clipping. Advantages are reduced morbidity, shorter hospital stay, and avoidance of major surgery. Complications vary by procedure but can include bleeding, perforation, and pancreatitis after ERCP.

\emph{Conclusion:} Therapeutic endoscopy is integral to modern surgery, offering minimally invasive options for a range of conditions.


\section{Documentation in surgery (July, 2006)}


\textbf{Question:} Explain the importance of proper documentation in surgical practice.

\textbf{Answer}

\emph{Introduction:} Meticulous documentation is crucial for patient safety, communication, and medico-legal protection.

\emph{Main Content:} Surgical documentation includes consent forms, operative notes, anesthesia records, and postoperative progress. Accurate records facilitate continuity of care among team members and provide a clear account of procedures performed, findings, and complications. They are vital in teaching, auditing outcomes, and defending against litigation. Electronic health records improve legibility and accessibility but require attention to data security. Timely completion with clear language is emphasized in Bailey \& Love.

\emph{Conclusion:} Comprehensive, timely notes are a professional responsibility that support effective clinical care and legal accountability.


\section{Role of ultrasonography in acute abdomen (January, 2006)}


\textbf{Question:} Describe how ultrasonography aids in evaluating the acute abdomen.

\textbf{Answer}

\emph{Introduction:} Ultrasound is a valuable first-line imaging modality for abdominal emergencies due to its portability and lack of radiation.

\emph{Main Content:} It quickly detects gallstones, biliary dilatation, appendiceal inflammation, and intra-abdominal fluid from trauma or ruptured ectopic pregnancy. In skilled hands, ultrasonography can differentiate inflammatory from obstructive causes of pain and guide percutaneous drainage of abscesses. It is particularly useful for bedside assessment in unstable patients or pregnant women. Limitations include operator dependence and reduced accuracy in the presence of bowel gas or obesity.

\emph{Conclusion:} Ultrasonography provides rapid, non-invasive information that guides management of the acute abdomen.


\section{Post operative pain management (January, 2006)}


\textbf{Question:} Outline methods for managing postoperative pain.

\textbf{Answer}

\emph{Introduction:} Effective control of postoperative pain improves patient comfort, facilitates breathing and mobilization, and reduces complications.

\emph{Main Content:} A multimodal approach is recommended. Systemic analgesics include opioids (morphine, fentanyl), non-opioid agents like paracetamol or NSAIDs, and adjuncts such as ketamine or gabapentinoids. Regional techniques—epidural or peripheral nerve blocks—provide excellent site-specific relief. Patient-controlled analgesia allows individualized dosing. Non-pharmacologic measures include ice, positioning, and psychological support. Regular assessment with pain scales guides dosing adjustments and helps avoid undertreatment or overdose.

\emph{Conclusion:} Tailoring analgesic strategies to each patient ensures optimal recovery and reduces postoperative morbidity.


\section{Parenteral nutrition (January, 2006)}


\textbf{Question:} Discuss indications and complications of parenteral nutrition.

\textbf{Answer}

\emph{Introduction:} Parenteral nutrition provides nutrients intravenously when enteral feeding is inadequate or impossible.

\emph{Main Content:} Indications overlap with those for TPN described in Question 21, including severe malabsorption, intestinal obstruction, and major abdominal trauma. Central venous access is usually required. Complications include catheter sepsis, thrombosis, electrolyte disturbances, and hepatic dysfunction. Meticulous sterile technique, monitoring of biochemical parameters, and careful tapering when transitioning to oral intake minimize risks.

\emph{Conclusion:} Parenteral nutrition is lifesaving in selected patients but demands close monitoring to avoid serious complications.


\section{Choice of suture materials (January, 2006)}


\textbf{Question:} What factors guide the choice of suture materials in surgery?

\textbf{Answer}

\emph{Introduction:} Selecting the appropriate suture is vital for optimal wound healing and depends on tissue type and required strength.

\emph{Main Content:} Sutures are classified as absorbable or non-absorbable, and natural or synthetic. Absorbable sutures like Vicryl or Polydioxanone are used in tissues that heal quickly, such as bowel or muscle. Non-absorbable materials (e.g., Nylon, Prolene) are preferred for skin closure or vascular anastomoses where long-term strength is needed. Factors influencing choice include tissue reactivity, handling properties, knot security, and the risk of infection. Surgeons also consider suture size and needle type for minimal trauma.

\emph{Conclusion:} Matching suture characteristics to the tissue and procedure promotes secure closure and minimizes complications.


\section{Hospital waste management (January, 2006)}


\textbf{Question:} Briefly describe the principles of hospital waste management.

\textbf{Answer}

\emph{Introduction:} Proper disposal of medical waste protects patients, staff, and the environment from infectious or hazardous materials.

\emph{Main Content:} Waste segregation is paramount. Sharp objects are placed in puncture-resistant containers, infectious waste in yellow bags, and non-infectious general waste in black bags, following local color coding. Chemical and pharmaceutical wastes require separate handling. Incineration, autoclaving, or chemical disinfection are common treatment methods. Staff training and use of personal protective equipment reduce risks of injury or contamination. Accurate record keeping ensures compliance with regulations.

\emph{Conclusion:} Adhering to waste management protocols safeguards health workers and communities while complying with environmental regulations.


\textbf{Conclusion}

These concise answers, based on standard surgical texts, cover common short-note topics encountered in MCPS examinations. Reviewing them as flashcards helps reinforce key principles and ensures a broad understanding of essential surgical concepts.

\end{document}
